\chapter{Scalars: Spin 0 Fields}
\section{Deducing Klein-Gordon Equation}
If we squared the operators in the original Schrodinger equation, we obtain
\begin{equation}
\left(i \hbar \frac{\partial}{\partial t}\right)\left(i \hbar \frac{\partial}{\partial t}\right) \phi=H^{2} \phi=\left(\mathbf{p}_{o p e r}^{2} c^{2}+m^{2} c^{4}\right) \phi
\end{equation}
which becomes
\begin{equation}
-\frac{\hbar^{2}}{c^{2}} \frac{\partial^{2}}{\partial t^{2}} \phi=\left(-\hbar^{2} \frac{\partial}{\partial X_{i}} \frac{\partial}{\partial X_{i}}+m^{2} c^{2}\right) \phi
\end{equation}
In a compact form, we have
\begin{equation}
-\frac{\partial}{\partial x^{0}} \frac{\partial}{\partial x_{0}} \phi=\left(\frac{\partial}{\partial x^{i}} \frac{\partial}{\partial x_{i}}+\frac{m^{2} c^{2}}{\hbar^{2}}\right) \phi
\end{equation}
where we define $\mu^2=\frac{m^{2} c^{2}}{\hbar^{2}}$. Re-arranging, we have the Klein-Gordon equation
\begin{qt}
    \begin{equation}
\left(\partial_{\mu} \partial^{\mu}+\mu^{2}\right) \phi=0
\end{equation}
The operation $\partial_{\mu} \partial^{\mu}=\partial^{\mu} \partial_{\mu}$ is called the \textbf{d'Alembertian} operator, and is the 4D Minkowski coordinates analogue of the 3D Laplacian operator of Cartesian coordinates.
\end{qt}
\subsection{The solutions to the Klein-Gordon Equation}
\begin{equation}
\phi(x)=\sum_{n=1}^{\infty} \frac{1}{\sqrt{2 V E_{n} / \hbar}}\left(A_{n} e^{-\frac{i}{\hbar}\left(E_{n} t-\mathbf{p}_{n} \cdot \mathbf{x}\right)}+\underbrace{B_{n}^{\dagger} e^{\frac{i}{\hbar}\left(E_{n} t-\mathbf{p}_{n} \cdot \mathbf{x}\right)}}_{\text {absent in NRQM }}\right)
\end{equation}
Because we are using the square of the relativistic Hamiltonian in RQM, \bluep{we get additional solutions of exponential form $+i(E_nt-\mathbf{p}_n\cdot\mathcal{x})/\hbar$ that also solve the relativistic Klein-Gordon equation}.

With an aim towards using natural units, we note the following relations, where wave number $k_{i}$ $=2 \pi / \lambda_{i}$ and we use the deBroglie relation $p^{i}=\hbar k^{i}$
\begin{qt}
    \begin{equation}
p_{\mu}=\left[\begin{array}{c}
{E / c} \\
{p_{i}}
\end{array}\right]=\left[\begin{array}{c}
{E / c} \\
{-p^{i}}
\end{array}\right]=\hbar k_{\mu}=\left[\begin{array}{c}
{\hbar \omega / c} \\
{-\hbar k^{\prime}}
\end{array}\right]
\end{equation}
in natural units
\begin{equation}
p_{\mu}=\left[\begin{array}{c}
{E} \\
{-p^{i}}
\end{array}\right]=k_{\mu}=\left[\begin{array}{c}
{\omega} \\
{-k^{i}}
\end{array}\right]
\end{equation}
\end{qt}
Recall the notation introduced in the previous chapter, we have
\begin{equation}
p x=p_{\mu} x^{\mu}=E t-p^{i} x^{i}=p^{\mu} x_{\mu}
\end{equation}
\begin{equation}
k x=k_{\mu} x^{\mu}=\omega t-k^{i} x^{i}=\frac{E t}{\hbar}-\frac{p^{i} x^{i}}{\hbar}=\frac{p_{\mu}}{\hbar} x^{\mu}
\end{equation}
In natural unit
\begin{equation}
E=\omega, \quad p_{i}=k_{i}, \quad p_{\mu}=k_{\mu}, \quad p x=k x
\end{equation}
\bluep{For free fields, a given wave with wave number vector $\mathbf{k}$ has a particular energy, and we can designate that energy via either $E_{\mathbf{k}} \text { or } \omega_{\mathbf{k}} .$ It is common practice for scalars to use $\mathbf{k} \text { (rather than } \mathbf{p})$ and $\mathbf{k}$ (rather than $E_{\mathbf{p}}$ or $E_{\mathbf{k}}$.)}

The Klein-Gordon equation solutions then become, in natural units
\begin{equation}
\phi(x)=\sum_{k} \frac{1}{\sqrt{2 V_{e l}}}\left(A_{k} e^{-i k x}+B_{k}^{\dagger} e^{i k x}\right)
\end{equation}
\begin{mybox}
In RQM, the solution $\phi$ is that of a general (sum of eigenstates) single particle state. Each eigenstate has mathematical form of
$$
\phi_{k, A}=\frac{e^{-i k x}}{\sqrt{V}}
$$
$$
\quad \phi_{\mathrm{k}, B^{\dagger}}=\frac{e^{i k x}}{\sqrt{V}}
$$
Each of these forms has what is called \textbf{unit norm}. That is, all such eigenstates are \textbf{orthonormal}:
\begin{equation}
\int \phi_{k, A}^{\dagger} \phi_{k^{\prime},A} d^{3} x=\frac{1}{V} \int e^{i k x} e^{-i k^{\prime} x} d^{3} x=\delta_{k k^{\prime}}
\end{equation}
Similar relations exist for $\phi_{k, B^{\dagger}}$.
\end{mybox}

\subsection{Deducing probability density in RQM}
Starting from the Klein-Gordon equation, first post-multiply it by $\phi^{\dagger}$, then subtract the complex conjugate equation post-multiplied by $\phi$,
\begin{equation}
\begin{array}{c}
{\left\{\frac{\partial^{2}}{\partial t^{2}} \phi=\left(\nabla^{2}-\mu^{2}\right) \phi\right\} \phi^{\dagger}} \\
{-\left\{\frac{\partial^{2}}{\partial t^{2}} \phi^{\dagger}=\left(\nabla^{2}-\mu^{2}\right) \phi^{\dagger}\right\} \phi}
\end{array}
\end{equation}
Since $\mu^{2} \phi^{\dagger} \phi-\mu^{2} \phi \phi^{\dagger}=0$, we obtain
\begin{equation}
i \frac{\partial}{\partial t}\left(\frac{\partial \phi}{\partial t} \phi^{\dagger}-\frac{\partial \phi^{\dagger}}{\partial t} \phi\right)=i \nabla \cdot\left((\nabla \phi) \phi^{\dagger}-\left(\nabla \phi^{\dagger}\right) \phi\right)
\end{equation}
where probability density and the probability current for a Klein-Gordon particle are
\begin{qt}
    \begin{equation}
\rho=j^{0}=i\left(\frac{\partial \phi}{\partial t} \phi^{\dagger}-\frac{\partial \phi^{\dagger}}{\partial t} \phi\right)
\end{equation}
and
\begin{equation}
\mathbf{j}=-i\left((\nabla \phi) \phi^{\dagger}-\left(\nabla \phi^{\dagger}\right) \phi\right) \quad j^{i}=-i\left(\phi_{,i} \phi^{\dagger}-\phi_{, i}^{\dagger} \phi\right)=i\left(\phi^{,i} \phi^{\dagger}-\phi^{\dagger, i} \phi\right)
\end{equation}
\end{qt}
Now we can define the \redp{4-currents} as
\begin{equation}
j^{\mu}=\left[\begin{array}{l}
{\rho} \\
{\mathbf{j}}
\end{array}\right]=\left[\begin{array}{l}
{\rho} \\
{j^{i}}
\end{array}\right]=\left[\begin{array}{l}
{j^{0}} \\
{j^{i}}
\end{array}\right]=i\left(\phi^{\mu} \phi^{\dagger}-\phi^{\dagger,\mu} \phi\right)
\end{equation}
The \redp{4D continuity equation} is then
\begin{qt}
    \begin{equation}
\frac{\partial j^{\mu}}{\partial x^{\mu}}=\partial_{\mu} j^{\mu}=j^{\mu}{ }_{,\mu}=0
\label{4d-continuity-eq}
\end{equation}
\end{qt}
(\ref{4d-continuity-eq}) tells us the important fact that \bluep{the 4-divergence of the 4-current of any conserved quantity is zero}.

\bluep{The total probability of unity is a relativistic invariant. Further $A_{k}$ here are constants that do not vary with frame.}\redp{So the probability of  finding any particular state is also independent of what frame the measurements are taken in.}

\subsection{Negative Energies in RQM}
If we apply traditional Hamiltonian operator $H$ as $i\partial/\partial t$ to $\phi_{\mathbf{k},B^{\dagger}}$:
\begin{equation}
i \frac{\partial \phi_{\mathbf{k}, B^{\dagger}}}{\partial t}=i \frac{\partial}{\partial t} \frac{e^{i k x}}{\sqrt{V}}=-\omega_{\mathbf{k}} \frac{e^{i k x}}{\sqrt{V}}=-\omega_{\mathbf{k}} \phi_{\mathbf{k}, B^{\dagger}}=E_{\mathbf{k}, B^{\prime}} \phi_{\mathbf{k}, B^{+}}
\end{equation}
Since $\omega_{\mathbf{k}}$ is always a positive number, we have states with \redp{negative energies in RQM, and we need QFT to solve this dilemma}. 

\section{Klein-Gordon Equation in Quantum Field Theory}
