\chapter{Interactions: The Underlying Theory}
\section{Interactions in RQM}
\subsection{Maxwell's equation with sources}
If we include the charge density and current density into Maxwell's equation, we have
\begin{equation}
\begin{aligned}
&\nabla \cdot \mathbf{E}=\rho_{charge}\\
&\nabla \times \mathbf{B}-\frac{\partial \mathbf{E}}{\partial t}=\mathbf{j}_{charge}\\
&\nabla \cdot \mathbf{B}=0\\
&\vec{\nabla} \times \mathbf{E}=-\frac{\partial \mathbf{B}}{\partial t}
\end{aligned}
\end{equation}
By introducing the potentials $\Phi$ and $\mathbf{A}$, where 
$$
\mathbf{B}=\nabla \times \mathbf{A}, \quad \mathbf{E}=-\nabla \Phi-\frac{\partial \mathbf{A}}{\partial t}
$$
one gets
\begin{equation}
\begin{aligned}
&-\nabla \frac{\partial \Phi}{\partial t}-\nabla(\nabla \cdot \mathbf{A})=\rho_{charge}\\
&\frac{\partial^{2} \mathbf{A}}{\partial t^{2}}-\nabla^{2} \mathbf{A}+\nabla \frac{\partial \Phi}{\partial t}+\nabla(\nabla\cdot\mathbf{A})=\mathbf{j}_{charge}
\end{aligned}
\end{equation}
Rewrite the equations above in terms of 4-potential and 4-current $-ej^{\mu}$=($\rho_{charge},\mathbf{j}_{charge}$):
\begin{equation}
    \partial^{\alpha}\partial_{\alpha}A^{\mu}(x)-\partial^{\mu}\left(\partial_{\nu}A^{\nu}(x)\right)=-ej^{\mu}(x)
\end{equation}
Using Lorenz gauge condition, we have
\begin{qt}
\begin{equation}
    \partial^{\alpha}\partial_{\alpha}A^{\mu}(x)=-ej^{\mu}
    \label{maxwell-4D-interaction-eqn}
\end{equation}
\end{qt}
\subsection{The classical Lagrangian density for interaction}
The full electromagnetic Lagrangian(density), including interactions, must give rise to (\ref{maxwell-4D-interaction-eqn}) when substituted into the Euler-Lagrange field equation:
$$
\frac{\partial}{\partial x^{v}}\left(\frac{\partial \mathcal{L}}{\partial \phi_{v}^{n}}\right)-\frac{\partial \mathcal{L}}{\partial \phi^{n}}=0, \quad \text { with } \quad \phi^{n}=A_{\mu} ; \mathcal{L}=\mathcal{L}^{e / m}
$$
One can prove that the full electromagnetic field classical Lagrangian is
\begin{qt}
\begin{equation}
\mathcal{L}^{e / m}=\underbrace{-\frac{1}{2}\left(\partial_{v} A_{\mu}(x)\right)\left(\partial^{v} A^{\mu}(x)\right)}_{\mathcal{L}_{0}^{e / m}}+ \underbrace{e j^{\mu}(x) A_{\mu}(x)}_{\mathcal{L}_{1}^{e / m}}
\label{full-electromagnetic-field-classical-lagrangian}
\end{equation}
where "0" and "1" subscripts denote the free and interaction parts, respectively, of the Lagrangian.
\end{qt}
\subsection{Electromagnetic Interactions in RQM}
Relations (\ref{maxwell-4D-interaction-eqn}) and (\ref{full-electromagnetic-field-classical-lagrangian}) hold for classical electromagnetism where $-ej^{\mu}$ is the classical electric charge 4-current. \textbf{In quantization, we assume the quantum form of the Lagrangian (density or total) is the same as the classical, and thus, so would be the resulting wave equation.} In RQM, we would then consider $A^{\mu}$ to represent the quantum photon state (ket, wave function). Thus, (\ref{full-electromagnetic-field-classical-lagrangian}) represents the RQM electromagnetic Lagrangian. But then, \bluep{how should one interpret $-ej^{\mu}$?} In Chap.3, we saw that the probability 4-current for an electron in RQM, where $\psi_{state}$ represents the electron wave function state, is
$$
j^{\mu}=(\rho, \mathbf{j})=\bar{\psi}_{\text {state}} \gamma^{\mu} \psi_{\text {state}} \quad \text { where } \partial_{\mu} j^{\mu}=0
$$
\redp{It seems natural to assume charge density varies directly with probability density.} Thus, we can assume
\begin{equation}
-e j^{\mu}=-e \bar{\psi}_{\text {state}} \gamma^{\mu} \psi_{\text {state}}
\label{electron-4-current-state}
\end{equation}
where the total charge of the electron would be
$$
-e \int j^{0} d^{3} x=-e
$$
Using (\ref{maxwell-4D-interaction-eqn}),(\ref{full-electromagnetic-field-classical-lagrangian}), and(\ref{electron-4-current-state}), we can then represent the \textbf{\redp{RQM interaction wave equation for a photon}} as
\begin{equation}
\partial^{\alpha} \partial_{\alpha} A_{\text {stale}}^{\mu}=-e \bar{\psi}_{\text {state}} \gamma^{\mu} \psi_{\text {state}}
\label{RQM-photon-interaction-wave-eqn}
\end{equation}
with the corresponding \textbf{\redp{RQM e/m interaction Lagrangian for a photon}} as
\begin{equation}
\mathcal{L}^{e / m}=\underbrace{-\frac{1}{2}\left(\partial_{v} A_{\mu,state}\right)\left(\partial^{v} A_{\text {state }}^{\mu}\right)}_{\mathcal{L}_0^{e/m}}+\underbrace{e \bar{\psi}_{\text {state}} \gamma^{\mu} \psi_{\text {state}} A_{\mu,state}}_{\mathcal{L}_I^{e/m}}
\label{RQM-em-interaction Lagrangian-for-photon}
\end{equation}
\redp{\textbf{(\ref{RQM-photon-interaction-wave-eqn}) governs the behavior of a photon ($A^{\mu}_{state}$) in the presence of an electron ($\psi_{state}$)}}.
\subsection{The electromagnetic interaction Dirac equation}
We now develop the full Dirac equation describing the electron interacting with a photon. To this end, consider that $\mathcal{L}_{0}^{e / m}$ of (\ref{RQM-em-interaction Lagrangian-for-photon}) represents the free photon part of the full e/m Lagrangian. If we assume $\mathcal{L}_{1}^{e / m}$ represents the e/m interaction part for both the photon and the electron, then \textbf{\redp{we need only add the free electron contribution to (\ref{RQM-em-interaction Lagrangian-for-photon}) to get a Lagrangian containing all terms relevant to photons, electrons, and interactions between them.}} Since from Chap 3
$$
\mathcal{L}_{0}^{1 / 2}=\bar{\psi}_{\text {state}}\left(i \gamma^{\mu} \partial_{\mu}-m\right) \psi_{\text {state}}
$$
Thus, the \textbf{full e/m Lagrangian} is
\begin{equation}
\mathcal{L}^{1 / 2,1}=\underbrace{-\frac{1}{2}\left(\partial_{v} A_{\mu,state}\right)\left(\partial^{v} A_{\text {state}}^{\mu}\right)}_{\mathcal{L}_{0}^{1}=\mathcal{L}_{0}^{e / m}}+\underbrace{\bar{\psi}_{\text {state}}\left(i \gamma^{\mu} \partial_{\mu}-m\right) \psi_{\text {state}}}_{\mathcal{L}_{0}^{1 / 2}}+\underbrace{e \bar{\psi}_{\text {state}} \gamma^{\mu} \psi_{\text {state}} A_{\mu,state}}_{\mathcal{L}_{1}^{1 / 2,1}=\mathcal{L}_{1}^{e/m}}
\label{full-e/m-lagrangian}
\end{equation}
To find the interaction form of the Dirac equation, we use (\ref{full-e/m-lagrangian}) in Euler-Lagrange equation with $\phi^{n=1}=\bar{\psi}$. The result is
\begin{qt}
\begin{equation}
\left(i \gamma^{\mu} \partial_{\mu}-m\right) \psi_{\text {state }}=-e \gamma^{\mu} \psi_{\text {state }} A_{\mu,state}
\label{full-Dirac-eqn}
\end{equation}
As an aside, using (\ref{full-e/m-lagrangian}) with $\phi^{n=2}=\psi$ results in the adjoint full Dirac equation.
\end{qt}

\begin{mybox}
A this point, some might be concerned that we have used the Lagrangian density methodology, which is normally reserved for quantum and classical fields, to develop the full Dirac equation for quantum states (corresponding to particles, not fields).One would expect to use the
total Lagrangian L (integration of $\mathcal{L}$ over all space) instead of $\mathcal{L}$, since (\ref{full-Dirac-eqn}) is a wave equation for interacting states, not fields.

However, even in the context of $1^{\text {st }}$ (particle) and $2^{\text {nd }}$ (field) quantization as we have come to understand them, the issue is not such a big one. This is because we did not employ commutation relations for fields $A^{\mu}$ and $\psi,$ analogous to Poisson bracket relations, in the above development, It is the adth $^{\mu} A^{\mu}$ and $\psi,$ analogous to Poisson bracket relations, in the above development, \textbf{it is the adoption of commutation relations for those fields that turns them into creation and destruction operators quantum mechanically.} We did not do that, so $A^{\mu}$ and $\psi$ remain as states, not quantum fields, in the above treatment. Of course, for RQM, we would still have commutation relations for dynamical variable gnerators, such as $p_{x}$ and $X_{1}$, though we would not have them for $A^{\mu}$ and $\psi$
\end{mybox}
\section{Interactions in QFT}
Our interacting spinor field and photon wave equations in the Heisenberg picture should simply be the following for fields,
\begin{equation}
\partial^{\alpha} \partial_{\alpha} A^{\mu}=-e \bar{\psi} \gamma^{\mu} \psi
\label{fields-interaction-wave-eqn-1}
\end{equation}
\begin{equation}
\left(i \gamma^{\mu} \partial_{\mu}-m\right) \psi=-e \gamma^{\mu} A_{\mu} \psi
\label{fields-interaction-wave-eqn-2}
\end{equation}
where the order of $\psi$ and $A_{\mu}$ in the equations above is unimportant, even though they are operators, since \redp{different type fields commute}. The associated Lagrangian for the $\psi$ and $A^{\mu}$ operator fields is
\begin{equation}
\mathcal{L}^{1 / 2,1}=\underbrace{-\frac{1}{2}\left(\partial_{v} A_{\mu}\right)\left(\partial^{v} A^{\mu}\right)}_{\mathcal{L}_{0}^{1}}+\underbrace{\bar{\psi}\left(i \gamma^{a} \partial_{\alpha}-m\right) \psi}_{\mathcal{L}_0^{1/2}}+\underbrace{e \bar{\psi} \gamma^{\mu} A_{\mu} \psi}_{\mathcal{L}_{I}^{1 / 2,1}}
\end{equation}

Modern day computers can help in providing numerical solutions to these equations, but early researchers in QFT did not have such things. Also, the route those researchers did take provides considerable insight into the inner workings of the theory. That route, for the QFT e/m interaction theory known as \textbf{quantum electrodynamics (QED)}, was forged in large part by Richard Feynman, Freeman Dyson, Julian Schwinger, and Sin-Itiro Tomonaga. It involves two things,
\begin{itemize}
    \item perturbation theory, and
    \item a trick known as the Interaction picture
\end{itemize}
\begin{qt}
The expectation value for any quantum field, including spinor and vector fields, is zero.
\end{qt}

\section{Interaction Picture}
It turns out that a third picture, the Interaction Picture (I.P.) is easier to use for interactions in
QFT. For one reason, it facilitates use of perturbation theory in place of trying to solve the coupled, non-linear, partial differential equations (\ref{fields-interaction-wave-eqn-1}) and (\ref{fields-interaction-wave-eqn-2}).

Additionally, the LP. allows us to analyze interacting fields using all the results of our free QFT development.

\underline{Breaking the Hamiltonian into Free and Interaction Parts}
The Hamiltonian (total, not density) for e/m interactions in the (Schrödinger Picture) S.P. can be expressed, from the Lagrangian density (\ref{full-e/m-lagrangian}) and the Legendre transformation, as (with $\phi^{r}$ generically representing any quantum field)
\begin{equation}
\underbrace{H^{S}}_{H}=\underbrace{H^{1 / 2,1}}_{\text {just e/m }}=\underbrace{\int\left(\pi_{r} \dot{\phi}^{r}-\mathcal{L}_{0}^{1}-\mathcal{L}_{0}^{1 / 2}\right) d^{3} x}_{H_{0}^{S}=H_{0}(\text { free part })}-\underbrace{\int \mathcal{L}_{I}^{1 / 2,1} d^{3} x}_{H_{I}^{S}(\text { interaction part })}
\end{equation}
Thus,$H=H_{0}+H_{I}^{S}$. Note, for future reference, that for all cases  $\mathcal{H}_{I}=-\mathcal{L}_{I}$ and $ H_{I}=-L_{I}$.

\underline{Using only the free part of the Hamiltonian to transform to the interaction picture}

The transformation from the S.P. to the I.P. is
\begin{equation}
    U_{0}=e^{-i H_{0} t}
\end{equation}
where $U_0$ is a unitary operator, where
\begin{equation}
U_{0}^{\dagger}|\Psi\rangle_{S}=|\Psi\rangle_{I}
\end{equation}
and where subscripts "S" and "I" on generic states $|\Psi\rangle$ indicate the S.P. and I.P, respectively. For operators, where superscripts "S" and "I" represent the S.P. and I.P., respectively,
\begin{equation}
U_{0}^{\dagger} \mathcal{O}^{S} U_{0}=\mathcal{O}^{I}
\end{equation}

\underline{Parts of the Hamiltonian expressed in the I.P.}

For the free part of the Hamiltonian operator $H_{0}=H_{0}^{S},$ we see that
\begin{equation}
H_{0}^{I}=U_{0}^{\dagger} H_{0}^{S} U_{0}=U_{0}^{\dagger} H_{0} U_{0}=e^{i H_{0} t} H_{0} e^{-i H_{0} t}=H_{0} e^{i H_{0} t} e^{-i H_{0} t}=H_{0}
\end{equation}
because $H_{0}$ commutes with itself. Thus,
\begin{equation}
H_{0}=H_{0}^{S}=H_{0}^{I}
\end{equation}
\bluep{this equality generally does not hold for the interaction part:}
$$
H_{I}^{I}=U_{0}^{\dagger} H_{I}^{S} U_{0} \neq H_{I}^{S}
$$
and thus, we will represent the interaction picture Hamiltonian as
\begin{equation}
H^{I}=H_{0}+H_{I}^{I}
\end{equation}
\subsection{Equations of Motion in the I.P.}
$$
\frac{d \mathcal{O}^{l}}{d t}=\frac{d}{d t}\left(U_{0}^{+} \mathcal{O}^{s} U_{0}\right)=\frac{d U_{0}^{\dagger}}{d t} \mathcal{O}^{s} U_{0}+\underbrace{U_{0}^{\dagger} \frac{\partial \mathcal{O}^{S}}{\partial t} U_{0}}_{\text {defined }=\frac{\partial \mathcal{O}^{I}}{\partial t}}+U_{0}^{\dagger} \mathcal{O}^{s} \frac{d U_{0}}{d t}=
$$
$$
\frac{d e^{i H_{0} t}}{d t} \mathcal{O}^{S} e^{-i H_{0} t}+e^{i H_{0} t} \mathcal{O}^{S} \frac{d e^{-i H_{0} t}}{d t}+\frac{\partial \mathcal{O}^{I}}{\partial t}=i H_{0} \underbrace{e^{i H_{0} t} \mathcal{O}^{S} e^{-i H_{0} t}}_{\mathcal{O}^{I}}-\underbrace{e^{i H_{0} t} \mathcal{O}^{S} e^{-i H_{0} t}}_{\mathcal{O}^{I}} i H_{0}+\frac{\partial \mathcal{O}^{I}}{\partial t}
$$
or
\begin{qt}
\begin{equation}
\frac{d \mathcal{O}^{I}}{d t}=-i\left[\mathcal{O}^{I}, H_{0}\right]+\frac{\partial \mathcal{O}^{I}}{\partial t}
\label{IP-operator-time-derivative}
\end{equation}
\end{qt}
where the last term is zero in this note, because we only deal with operators for which $\partial \mathcal{O}^{S} / \partial t=0$. \redp{Thus, the equation of motion for operators in the I.P. depends only on the free part of the Hamiltonian.}

The \underline{I.P. equation of motion for states is}
\begin{equation}
i \frac{d}{d t}|\Psi\rangle_{I}=H_{I}^{I}|\Psi\rangle_{I}
\label{IP-eq-of-motion-state}
\end{equation}
\redp{and hence, the equation of motion for states in the I.P. depends only on the interaction part of the Hamiltonian.}

\underline{The I.P. equation of motion for expectation values is}
\begin{equation}
\frac{d \overline{\mathcal{O}}}{d t}={}_{I}\left\langle\Psi\left|\left(-i\left[\mathcal{O}^{I}, H^{I}\right]+\frac{\partial \mathcal{O}^{I}}{\partial t}\right)\right| \Psi\right\rangle_{I}
\end{equation}
\begin{mybox}
Note that (\ref{IP-operator-time-derivative}) has the same form as the operator equation of motion in the H.P., except that we have $H_0$ in the I.P. and H in the H.P.\textbf{ Hence, we can teke all results we obtained for operator behavior in the H.P. free field and use them in the I.P. for interacting fields.}

For field operators such as $\mathcal{O}^{l}=\phi,\psi, \text { or } A^{\mu})$, (\ref{IP-operator-time-derivative}) has identical form to the H.P. equation of motion for fields where $H=H_0$. Thus (\ref{IP-operator-time-derivative}) reduces to the Klein-Gordon equation for scalars, the free Dirac equation for spinor fields, and the free Maxwell equation for photons. \textbf{Quantum fields in the I.P. behave just like the free quantum fields}
\end{mybox}
\subsection{Visualizing states in the I.P.}
Consider a single  particle scalar plane wave state expressed in the coordinate basis. In the S.P., it looks like, where K is a normalization factor and sub/superscript meaning should be obvious,
$$
|\phi\rangle_{S}=\phi_{\text {state }}^{S}=K e^{-i E t+i \mathbf{k} \cdot \mathbf{x}}=K e^{-i E_{0} t-i E_{I} t+i \mathbf{k} \cdot \mathbf{x}}
$$
Note that $E_{t}$ is a number, and thus is the same in any picture. Transform the state to the I.P.,
$$
U_{0}^{\dagger}|\phi\rangle_{S}=e^{i H_{0} t} \phi_{S, \text { state }}=K e^{i H_{0} t} e^{-i E_{0} t-i E_{I} t+i k \cdot x}=K e^{i E_{0} t} e^{-i E_{0} t-i E_{I} t+i \mathbf{k} \cdot \mathbf{x}}=K e^{-i E_{I} t+i \mathbf{k} \cdot \mathbf{x}}
$$
So we see that \redp{the state in the I.P. varies in time only with the interaction energy}. The operator $U_{0}^{\dagger}$ takes out the $H_{0}$ dependence of the ket.

In the I.P.
\begin{itemize}
    \item the state equation of motion depends on only the interaction Hamiltonian $H_I^I$, \item operator equations of motion depend on only the free Hamiltonian $H_{0},$ thus, importantly, 
    \item the operator equations of motion in the I.P. are the same as the operator equations of motion in the H.P. for free fields (i.e., for $\left.H^{H}=H_{0} \text { with } H_{I}^{H}=0\right),$ so all operator relations derived for free field are valid in I.P.
    \item meaning the free field case Klein-Gordon, Dirac, and Maxwell equations (of motion) from the H.P. are the same as those in the interacting case in the I.P., and so
    \item quantum fields $\phi, \psi,$ and $A^{\mu}$ in the L.P. (the solutions to the field equations of motion) are the same as the free quantum fields solutions in the H.P.
\end{itemize}
So in the I.P. \redp{We only need to solve the state equation of motion(\ref{IP-eq-of-motion-state}).}
\section{S matrix}
\redp{Each $S_{f i}$ of $S$ Matrix is the transition amplitude between an initial eigenstate and a final eigenstate, $S_{f i}^{2}$ is probability of transition from initial eigenstate $|i\rangle$ to final eigenstate $|f\rangle$}. For example
\begin{center}
    $S_{21}^{\dagger} S_{21}=\left|S_{21}\right|^{2}=$ probability of 1 st eigenstate transitioning to $2 \mathrm{nd}$
\end{center}
$S_{fi}$ is a transition amplitude for a particular reaction. That is, for the operator $S_{oper,fi}$
$$
S_{fi}=\left\langle f\left|S_{\text {oper}, fi}\right| i\right\rangle
$$
\begin{equation}
\left.S_{\text {oper}, f i}|i\rangle= S_{f i}|f\rangle \quad \text { (no sum on } i \text { or } f\right)
\end{equation}
and
\begin{equation}
\sum_{f}\left|S_{fi}\right|^{2}=1
\end{equation}
in general
\begin{qt}
\begin{equation}
S_{f i}=\left\langle f\left|S_{o p e r}\right| i\right\rangle
\end{equation}
\end{qt}
\section{Finding the S operator}
We can find the $S_{\text {oper }}$ from the state equation of motion, the only thing we haven't already solved for in the I.P. formulation. In the I.P., our state equation of motion, where $\mathbf{\Psi}$ ) represents a generic state (multiparticle typically), is
$$
i \frac{d}{d t}|\Psi(t)\rangle_{I}=H_{I}^{I}|\Psi(t)\rangle_{I}
$$
Let
\begin{equation}
|F\rangle=\sum_{f} S_{f i}|f\rangle=\left|\Psi\left(t_{f}\right)\right\rangle_{I}=S_{o p e r}\left(t_{f}, t_{i}\right)\left|\Psi\left(t_{i}\right)\right\rangle_{I}=S_{o p e r}\left(t_{f}, t_{i}\right)|i\rangle
\end{equation}
Taking our final time $t_f$ as time $t$ in the eqn. of motion, we have
\begin{equation}
|\Psi(t)\rangle_{I}=S_{\text {oper}}\left(t, t_{i}\right)\left|\Psi\left(t_{i}\right)\right\rangle_{I}
\end{equation}
Using the equation  above in the equation of motion for state yields
\begin{equation}
i \frac{d}{d t}\left(S_{o p e r}\left|\Psi\left(t_{i}\right)\right\rangle_{I}\right)=H_{l}^{I}\left(S_{o p e r}\left|\Psi\left(t_{i}\right)\right\rangle_{I}\right)
\end{equation}
This becomes
\begin{equation}
i \frac{d S_{\text {oper }}}{d t}\left|\Psi\left(t_{i}\right)\right\rangle_{I}+i S_{\text {oper }}\underbrace{ \frac{d}{d t}\left|\Psi\left(t_{i}\right)\right\rangle_{I}}_{=0\text{ as indep. of }t_f}=H_{I}^{I} S_{\text {oper }}\left|\Psi\left(t_{i}\right)\right\rangle_{I}
\end{equation}
\begin{qt}
and thus the differential equation for $S_{oper}$
\begin{equation}
i \frac{d S_{\text {oper}}}{d t}=H_{I}^{I} S_{\text {oper}}
\end{equation}
This has the solution
\begin{equation}
S_{\text {oper}}=e^{-i \int_{t_{i}}^{t_{f}} H_{I}^{I} d t}=e^{-i \int_{t_{i}}^{I f} \int_{V} \mathcal{H}_{I}^{I} d^{4} x}
\end{equation}
In infinite volume and time frame,
\begin{equation}
S=S_{\text {oper }}\left(V\rightarrow\infty t_{f} \rightarrow \infty, t_{i} \rightarrow-\infty\right)=e^{-i \int_{-\infty}^{\infty} \mathcal{H}_{I}^{I} d^{4} x}
\label{S-oper}
\end{equation}
\end{qt}
\begin{mybox}
In Fock space where every eigenstate can be visualized as a separate axis in an infinite dimensional space, the $S_{oper}$ can be visualized as a sort of abstract "rotation" in that space. \textbf{The initial state vector $|i\rangle$ is "rotated" by the $S_{oper}$ into a new vector with components along the eigenstate basis axes.}
\end{mybox}
\section{Expanding S operator}
$S_{oper}$ can be expanded (in a Taylor series like $e^{x}=1+x+x^{2} / 2 !+x^{3} / 3 !+\ldots$ ) as
$$
S_{o p e r}\left(t_{f}, t_{i}\right)=e^{-i \int_{t_{i}}^{t_{f}} H_{I}^{I}(t) d t}=\sum_{n=0}^{\infty} \frac{(-i)^{n}}{n !} \int_{t_{i}}^{t_{f}} \dots \int_{t_{i}}^{t_{f}} T\left\{H_{I}^{I}\left(t_{1}\right) H_{I}^{I}\left(t_{2}\right) \dots H_{I}^{I}\left(t_{n}\right)\right\} d t_{1} d t_{2} \dots d t_{n}
$$
If the $H^{I}$ above were numeric functions of time, it wouldn't really matter what order, with respect to the $t_{n}$, we carry out the integrations above. However, since they are comprised of operators that act on a ket state to their right, we have to be sure that at each point in the integration, the time-wise earliest operators are acting first.

This means that at each point in the $n$ dimensional space each axis is a different $t_{n}$, the $H_{I}^{I}$ dependent on the earliest time of the $t_{n}$ should act first, the $H_{I}^{I}$ dependent on the next earliest of the $t_{n}$ should act next. \textbf{The e order of the integrand operators is rearranged as we integrate over all $t_n$ dimensions, such taht the operator are time ordered at every point ($t_1,t_2,t_3,...$)}

Taking our integration limits to infinity in both space and time into the time ordered infinite spacetime $S_{\text {oper }},$ i.e., the \textbf{Dyson expansion of the S operator},
\begin{qt}
\begin{equation}
S=\sum_{n=0}^{\infty} \frac{(-i)^{n}}{n !} \int_{-\infty}^{\infty} \ldots . \int_{-\infty}^{\infty} T\left\{\mathcal{H}_{I}^{I}\left(t_{1}\right) \mathcal{H}_{I}^{I}\left(t_{2}\right) \ldots \mathcal{H}_{I}^{I}\left(t_{n}\right)\right\} d^{4} x_{1} d^{4} x_{2} \ldots d^{4} x_{n}
\end{equation}
Also note the symbols we can use for the terms above as
$$
S=\underbrace{I}_{S^{(0)}}\underbrace{-i \int_{-\infty}^{\infty} \mathcal{H}_{I}^{I}\left(x_{1}\right) d^{4} x_{1}}_{S^{(1)}}\underbrace{-\frac{1}{2!} \int_{-\infty}^{\infty} \int_{-\infty}^{\infty} T\left\{\mathcal{H}_{I}^{I}\left(x_{1}\right) \mathcal{H}_{I}^{I}\left(x_{2}\right)\right\} d^{4} x_{1} d^{4} x_{2}}_{S^{(2)}}+...=\sum_{n=0}^{\infty} S^{(n)}
$$
\end{qt}

\section{Wick's Theorem Applied to Dyson Expansion}
Time ordering is cumbersome to handle, because \textbf{we can't keep the same order of operators throughout the integration over time}. Fortunately, we can convert $S$ into non-time ordered form, where the order of operators does not change during integration, via a handy theorem developed by Gian-Carlo Wick.

\redp{Wick's theorem converts time ordered products of operators into normal ordered t products of operators and some things called "contractions".}
\begin{qt}
For generic fields $A$ and $B$ (either of which could be $\phi, \psi, A^{\mu}, \phi,$ etc) where
$$
A=\underbrace{A^{+}}_{\text {destruc }}+\underbrace{A^{-}}_{\text {creation }}=A^{d}+A^{c}
$$
$$
B=\underbrace{B^{+}}_{\text {destruc }}+\underbrace{B^{-}}_{\text {creation }}=B^{d}+B^{c}
$$
A contraction is defined as (note the under bracket symbol)
\begin{equation}
\underbracket{A\left(x_{1}\right) B}\left(x_{2}\right)=\left[A_{1}^{+}\left(x_{1}\right), B^{-}\left(x_{2}\right)\right]_{\mp}=\left[A^{d}\left(x_{1}\right), B^{c}\left(x_{2}\right)\right]_{\mp} \quad \text { if } t_{2}<t_{1}
\end{equation}
\begin{equation}
=\pm\left[B^{+}\left(x_{2}\right), A^{-} \left(x_{1}\right)\right]_{\mp}=\pm\left[B^{d}\left(x_{2}\right), A^{c}\left(x_{1}\right)\right]_{\mp} \quad \text { if } t_{1}<t_{2}
\label{contraction}
\end{equation}
In these two cases, we have two sets of time+location coordinates $(t_1,x_1)$ and $(t_2,x_2)$. The plus sign subscript implies anti-commutation, which is used \textbf{is both A and B are fermions.}$\pm$ in front of the relations on the 2 nd row takes a "+" sign for commutation, "-" for anti-commutation. \redp{All commutators/anti-commutators are zero unless $A=\phi$ and $B=\phi^{\dagger}$, or $A=\psi$ and $B=\bar{\psi}$.}
\end{qt}
Note that (\ref{contraction}) has the same form as the Feynmann propagator. Thus, whenever a contraction of scalar field is non-zero, it is a scalar Feynmann propagator. Similar results hold for Spinor and photon fields. Thus, \bluep{special cases of contractions (the only non-zero cases) are}
$$
\underbracket{\phi\left(x_{1}\right) \phi^{\dagger}}\left(x_{2}\right)=\underbracket{\phi^{\dagger}\left(x_{2}\right) \phi}\left(x_{1}\right)=i \Delta_{F}\left(x_{1}-x_{2}\right)
$$
Remember, the propagator is a number because of the commutation relations like $[a(k),a^{\dagger}(k)]=[b(k),b^{\dagger}(k)]=1$.
$$
\underbracket{\psi_{\alpha}\left(x_{1}\right) \bar{\psi}}_{\beta}\left(x_{2}\right)=-\underbracket{\bar{\psi}_{\beta}\left(x_{2}\right) \psi}_{\alpha}\left(x_{1}\right)=i S_{F \alpha \beta}\left(x_{1}-x_{2}\right)
$$
$$
\underbracket{A^{\mu}\left(x_{1}\right) A}^{\nu}\left(x_{2}\right)=i D_{F}^{\mu \nu}\left(x_{1}-x_{2}\right)
$$
\subsection{Review of normal ordering}
Normal ordering consists of placing construction operators to the right-hand side inside a given term.
\begin{center}
    $N(A B C D \ldots)=$ all destruction operators placed to right of all creation operators.
\end{center}
As part of our definition of normal ordering, note that the \textbf{switching places of two adjacent fermionic fields gives rise to a sign change}. This can be justified, in part, because fermionic fields obey anti-commutation relations. For example $^{1},$ given $[C, D]_{+}=0, C D=-D C$
\begin{qt}
Normal ordering of terms including contractions:

For B a destruction operator
\begin{equation}
N\left\{\underbracket{A\left(x_{1}\right) B^{d}\left(x_{2}\right) C}\left(x_{3}\right)\right\}=\pm \underbracket{A\left(x_{1}\right) C}\left(x_{3}\right) B^{d}\left(x_{2}\right)
\end{equation}
where minus sign for C and $B^{d}$ both fermionic. And for B a creation operator
\begin{equation}
N\left\{\underbracket{A\left(x_{1}\right) B^{c}\left(x_{2}\right) C}\left(x_{3}\right)\right\}=\pm B^{c}\left(x_{2}\right) \underbracket{A\left(x_{1}\right) C}\left(x_{3}\right)
\end{equation}
\end{qt}
\begin{mybox}
We will henceforth use subscripts in place of parenthesis arguments for fields, i.e.,
$$
A\left(x_{1}\right) \rightarrow A_{x_{1}} \quad A\left(x_{1}\right) B\left(x_{1}\right) C\left(x_{1}\right) \rightarrow(A B C)_{x_{1}}
$$
$$
\psi\left(x_{1}\right) \gamma^{\mu} A_{\mu}\left(x_{1}\right) \bar{\psi}\left(x_{1}\right) \rightarrow\left(\psi \gamma^{\mu} A_{\mu} \bar{\psi}\right)_{x_{1}}
$$
\end{mybox}
\subsection{Wick's theorem}
We only state Wick's theorem here and leave the justification to other textbook. The theorem turns a time ordered product into a series of normal ordered terms and contractions, which, as we noted, helps us in calculating interaction probabilities, because we can keep the operators in the same order as we integrate over time.
\begin{qt}
$$
T\left\{(A B \ldots)_{x_{1}} \ldots \ldots(A B \ldots)_{x_{n}}\right\}=N\left\{(A B \ldots)_{x_{1}} \ldots \ldots .(A B \ldots)_{x_{n}}\right\}
$$
$$
+N\left\{(\underbracket{A B . .) x_{1}(A} B .) x_{2} \cdots\right\}+N\left\{(\underbracket{A B . .) x_{1}(A B} .) x_{2} \cdots\right\}+\dots
$$
\begin{equation}
    +N\left\{(\overunderbraces{&\br{2}{A-A-contraction}}{&A&B C...) x_{1}(&AB}{&&\br{2}{B-B-contraction}}C...) x_{2} \cdots\right\}+N\left\{(\overunderbraces{&\br{2}{A-A-contraction}}{&A&BC.. .) x_{1}(&ABC}{&&\br{2}{B-C-contraction}}...) x_{2} \cdots\right\}+etc
\label{wick-theorem}
\end{equation}
Note that there are no contractions between operators operating-at the same time. We say there
are no "equal times contractions" in Wick's theorem. \redp{Equal-time contractions don't play a role in Wick's theorem for QED}.
\end{qt}

\section{Wick's Theorem in Words}
To get Wick's theorem, we start with a series of operator fields, operating in arbitrary order and
set it equal to itself, i.e., $A_{1} B_{2} C_{3} D_{4} \ldots=A_{1} B_{2} C_{3} D_{4} \ldots$

On the LHS, we then re-arrange operator fields using commutation/ant-commutation relations such that earlier times are to the right of later times. We herein use the symbol $T_{c}$ to represent this re-ordering procedure.  The final result of the LHS equals the original LHS expression, since at each, we simply substituted equivalent relations for the original pair of adjacent operators.

On the RHS, we re-arrange operator fields using commutation/anti-commutation relations such that destruction are all to the right of creation operators. We herein use the symbol $N_{c}$ to represent this re-ordering procedure. The final result of the RHS equals the original RHS expression.

The final result of these operations is the same as employing Wick's theorem.

\section{Comment on Normal Ordering of the Hamiltonian Density}
Non-zero commulators (anti-commulators) values are very small, $\approx0$ at macroscopic scales. So from human perspective, all fields effectively commute(anti-commute). So $\mathcal{H}$ could be normal ordered at microscale, but our classical theory formulation would be blind to it \& have evolved in non-normal ordered from.