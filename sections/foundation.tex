\chapter{Foundations}
\section{Natural Units and Dimensions}
Convenient systems of units start with arbitrary definitions for units of certain fundamental. quantities and derive the remaining units from laws of nature. To see how this works, assume we know three basic laws of nature and we want to devise a system of units from scratch. We will do this first for the cgs system and then for natural units.
The three lawas are:
\begin{qt}
\begin{itemize}
    \item The distance $L$ traveled by a photon is the speed of light multiplied by its time of travel. $L=c t$
    \item The energy of a massive particle is equal to its mass (at rest) $m$ times the speed of light squared. $E=mc^2$
    \item The energy of a photon is proportional to its frequency $f$. The constant of proportionality is Planck's constant $h . E=h f$ or re-expressed as $E=\hbar \omega$
\end{itemize}
\end{qt}
In natural units:\redp{the $c$ and $\hbar$ are dimensionless and equal to 1. The unit for energy is $MeV$}.

\section{Notation}
We shall use a notation defining \textbf{contravariant components $x^{\mu}$ of the $4 \mathrm{D}$ position vector} as $3 \mathrm{D}$ Cartesian coordinates $X_{i}$ plus $c t$ (see Appendix $\bar{A}$ if you are not comfortable with this), i.e.,
\begin{equation}
x^{\mu}=\left[\begin{array}{l}
{x^{0}} \\
{x^{1}} \\
{x^{2}} \\
{x^{3}}
\end{array}\right]=\left[\begin{array}{l}
{c t} \\
{X_{1}} \\
{X_{2}} \\
{X_{3}}
\end{array}\right]=\left[\begin{array}{l}
{c t, X_{i}}
\end{array}\right]^{T}
\end{equation}
From special relativity, we know the differential proper time passed on an object (with $c=1$ ) is
\begin{equation}
(d \tau)^{2}=(d t)^{2}-d X_{1} d X_{1}-d X_{2} d X_{2}-d X_{3} d X_{3}
\end{equation}
If we define \textbf{covariant components} of the 4D position vector as
\begin{equation}
x_{\mu}=\left[\begin{array}{l}
{x_{0}} \\
{x_{1}} \\
{x_{2}} \\
{x_{3}}
\end{array}\right]=\left[\begin{array}{c}
{t} \\
{-X_{1}} \\
{-X_{2}} \\
{-X_{3}}
\end{array}\right]=\left[t,-X_{i}\right]^{T}
\end{equation}
then
\begin{equation}
(d \tau)^{2}=d x^{0} d x_{0}+d x^{1} d x_{1}+d x^{2} d x_{2}+d x^{3} d x_{3}=d x^{\mu} d x_{\mu}
\end{equation}
Using metric tensor $g_{\mu\nu}$, we have the following relation:
\begin{equation}
x_{\mu}=g_{\mu \nu} x^{\nu}=\left[\begin{array}{cccc}
{1} & {0} & {0} & {0} \\
{0} & {-1} & {0} & {0} \\
{0} & {0} & {-1} & {0} \\
{0} & {0} & {0} & {-1}
\end{array}\right]\left[\begin{array}{c}
{x^{0}} \\
{x^{1}} \\
{x^{2}} \\
{x^{3}}
\end{array}\right]
\end{equation}
\redp{The inverse of $g_{\mu\nu}$, $g^{\mu\nu}$, has the exact same form.}Thus,
\begin{equation}
(d \tau)^{2}=g_{\mu v} d x^{\mu} d x^{v}=g^{\mu v} d x_{\mu} d x_{v}
\end{equation}
\begin{qt}
Partial derivative w.r.t. $x^{\mu}$ and $x_{\mu}$ are:
\begin{equation}
\partial_{\mu}=\frac{\partial}{\partial x^{\mu}}=\left(\frac{\partial}{\partial t}, \frac{\partial}{\partial x^{i}}\right)^{T}=\left(\frac{\partial}{\partial t} \cdot \frac{\partial}{\partial X_{i}}\right)^{T}
\end{equation}
and
\begin{equation}
\partial^{\mu}=\frac{\partial}{\partial x_{\mu}}=\left(\frac{\partial}{\partial t}, \frac{\partial}{\partial x_{i}}\right)^{T}=\left(\frac{\partial}{\partial t},-\frac{\partial}{\partial X_{i}}\right)^{T}
\end{equation}
\end{qt}
For a matrix, we can raise the index as
\begin{equation}
M^{\mu \nu}=g^{\mu \alpha} g^{\nu \beta} M_{\alpha \beta}
\end{equation}
\section{Review of Variational Methods}
Recall also, that given the Lagrangian, we could find the Hamiltonian $H,$ via the Legendre transformation (employing a Cartesian system where $x^{i}=x_{i}$ and $p^{i}=p_{i}$),
\begin{equation}
H=p_{i} \dot{x}^{i}-L, \quad \text { where } p_{i}=\frac{\partial L}{\partial \dot{x}^{i}}=m \dot{x}^{i}\left(=p^{\prime} \text { for Cartesian system }\right)
\end{equation}
$p_{i}$ is the conjugate, or canonical, momentum of $x$ '. (Note that a contravariant component in the denominator is effectively equivalent to a covariant component in the entire entity, and vice versa.) Hence, we can define:
\begin{qt}
\textbf{First quantization}

Keeping the classical Hamiltonian and, changing Poisson brackets to commutators, we could just as readily have used the Lagrangian L, or the equations of motion instead.
\end{qt}
\subsection{Classical Field Theory}
From particle theory to field theory, we have the following things changed:
\begin{qt}
$$L, H, e t c \rightarrow \mathcal{L}, \mathcal{H},$$ 
\[
x^{i}(t) \rightarrow \phi^{r}\left(x^{\mu}\right)
\]
$$t \rightarrow x^{\mu}$$
\end{qt}
Classical field theory is analogous in many ways to classical particle Lagrangian $L$, we have the Lagrangian density $\mathcal{L}$. Instead of time $t$ as an independent variable, we have $x^{\mu}=x^{0}, x^{1}, x^{2}, x^{3}=t, {x^{i}}$ as independent variables. Instead of a particle described by $x^{i}(t),$ \textbf{we have a field value described by $\phi(x^{\mu})$, where $r$ designates different field types, or possibly, different spatial components of the same vector field.}
\begin{qt}
The Euler-Lagrange equation for fields becomes
\begin{equation}
\frac{\partial}{\partial x^{\mu}}\left(\frac{\partial \mathcal{L}}{\partial \phi^{r}, \mu}\right)-\frac{\partial \mathcal{L}}{\partial \phi^{r}}=0
\end{equation}
and
\begin{equation}
\mathcal{H}=\pi_{r} \dot{\phi}^{r}-\mathcal{L}, \quad \text { where } \pi_{r}=\frac{\partial \mathcal{L}}{\partial \dot{\phi}^{r}}
\end{equation}
with $\pi_r$ being the conjugate momentum density of the field $\phi^r$. And the action is
\begin{equation}
S=\int_{T} \int_{V} \mathcal{L}\left(\phi, \phi_{,\mu}\right) d^{3} \mathbf{x} d t=\int_{\Omega} \mathcal{L}\left(\phi, \phi_{, \mu}\right) d^{4} x
\end{equation}
\end{qt}
\subsection{Key concepts in field theory}
For fields
\begin{equation}
\frac{\partial \phi}{\partial t}=\frac{d \phi}{d t}=\dot{\phi}
\end{equation}
This is generally not true for other quantities. For fields, 
\begin{equation}
\frac{d \phi}{d t}=\frac{\partial \phi}{\partial x^{\prime}} \frac{d x^{\prime}}{d t}+\frac{\partial \phi}{\partial t} \frac{d t}{d t}
\end{equation}
\redp{where $\frac{dx^i}{dt}$ are equal to zero}.

For a single particle, particle position coordinates are the generalized coordinates and particle momentum components are its conjugate momenta. For fields, each field is itself a generalized coordinate and each field has its own conjugate momentum (density). As noted, this field conjugate momentum (density) is different from the physical momentum (density) that the field possesses.

For conjugate and physical momentum densities, we have the following density relation
$$
p_{t}=\frac{\partial L}{\partial \dot{x}^{t}} \quad \frac{\text { for small particle in medium, }}{\text { divide by particle volume }}, \quad \mathcal{R}_{i}=\frac{\partial \mathcal{L}}{\partial \dot{x}^{i}}
$$
We note carefully that our $x^i$ here is the position coordinate of a point fixed relative to the field and thus is time dependent. Further, the total derivative $\dot{x}^i$ equals the partial derivative w.r.t. time, since $x^i$ in the present case is only a function of time. Now take the \redp{conjugate momentum density relation for relativistic fields},
$$
\pi_{r}=\frac{\partial \mathcal{L}}{\partial \dot{\phi}^{r}}
$$
and
$$
\frac{\mathcal{R}_{i}}{\pi_{r}}=\frac{\partial \mathcal{L} / \partial \dot{x}^{i}}{\partial \mathcal{L} / \partial \dot{\phi}^{r}}=\frac{\partial \dot{\phi}^{r}}{\partial \dot{x}^{\prime}}=\frac{\partial \phi^{r} / \partial t}{\partial x^{i} / \partial t}=\frac{\partial \phi^{r}}{\partial x^{i}} \quad \rightarrow \quad \mathcal{R}_{i}=\pi_{r} \frac{\partial \phi^{r}}{\partial x^{i}} \quad \rightarrow \quad \mathcal{R}^{i}=-\pi_{r} \frac{\partial \phi^{r}}{\partial x^{i}}
$$
The partial derivative of $\phi^{r}$ with respect to either of our definitions of $x^{i}$ (time dependent as the moving position of a point fixed to the field, or time independent as coordinates fixed in space) is the same because by definition, partial derivative means we hold everything else here constant, Thus, the above relation holds in field theory when we consider the $x^{i}$ as independent variables (coordinates fixed in space).

The field equation (equations of motion) for relativistic fields keep the exact same form in any inertial frame of reference, i.e., they are \redp{Lorentz invariant}. Four scalars(world scalars) are \redp{invariant under a Lorentz transformation and look exactly the same to any observer.}
\begin{qt}
The mass $m$ in relativity of a free particle is four scalar, where $m^2=p^{\mu}p_{\mu}$.

Demanding that the Euler-Lagrange equation be Lorentz invariant, we know that, within that equation, $x^{\mu}$,$\phi^r$, and derivatives of $x^{\mu}$ are Lorentz covariant or invariant. \redp{So, in order for the whole equation to be Lorentz invariant, the Lagrangian density $\mathcal{L}$ must be invariant.}

Further, L, H, and $\mathcal{H}$ are not Lorentz scalars.
\end{qt}

\section{Schrodinger vs. Heisenberg Pictures}
In Schrodinger picture, the expectation value is calculated as:
$$
\overline{\mathcal{O}}=\int \psi^{\dagger} \mathcal{O} \psi d^{3} x=\langle\psi|\mathcal{O}| \psi\rangle
$$
The time derivative of the expectation value is
\begin{equation}
\frac{d \bar{\mathcal{O}}}{d t}=\frac{d}{d t}\langle\psi|\mathcal{O}| \psi\rangle=\left\langle\frac{\partial \psi}{\partial t}|\mathcal{O}| \psi\right\rangle+\left\langle\psi\left|\frac{\partial \mathcal{O}}{\partial t}\right| \psi\right\rangle+\left\langle\psi|\mathcal{O}| \frac{\partial \psi}{\partial t}\right\rangle
\end{equation}
where we use partial derivative in the brackets because our states and operators are functions of $x^i$ and $t$.

In the Schródinger picture (S.P.), the solutions to the Schrödinger equation

\begin{equation}
i \frac{\partial \psi_{S}}{\partial t}=H \psi_{S} \quad \text { or } \quad i \frac{\partial}{\partial t}|\psi\rangle_{S}=H|\psi\rangle_{S}
\label{schrodinger-equation}
\end{equation}
\redp{In the Schrodinger picture, the operators are usually not time dependent.} For example, using the familiar momentum operator $p_1^S=i\partial/\partial x^1$ for the S.P. in the $x^1$ direction, with 
$$
\psi_{S}=A e^{-i\left(E t-p\cdot x\right)}=|\psi\rangle_{S} \quad A^{\dagger} A=\frac{1}{V}
$$
we have
$$
\begin{array}{l}
{\bar{p}_{1}=\int A^{+} e^{i\left( Et-p^{\prime}x^{\prime}\right)}\left(i \frac{\partial}{\partial x^{1}}\right) A e^{-i\left(Et-p^{\prime}x^{\prime}\right)} d^{3} x={}_{s}\langle\psi|p_{1}^{S}| \psi\rangle_{s}}
\end{array}
$$
Since there is no $t$ in the operator, we have
$$
\frac{d p_{1}^{S}}{d t}=\frac{\partial p_{1}^{S}}{\partial t}=0
$$
and
\begin{equation}
\frac{d \bar{p}_{1}}{d t}=\frac{d}{d t}\left\langle\psi\left|p_{1}^{s}\right| \psi\right\rangle=\left\langle\frac{\partial \psi}{\partial t}\left|p_{1}^{s}\right| \psi\right\rangle_{s}+s\left\langle\psi\left|\frac{\partial p_{1}^{s}}{\partial t}\right| \psi\right\rangle_{s}+s\left\langle\psi\left|p_{1}^{s}\right| \frac{\partial \psi}{\partial t}\right\rangle_{s}
\end{equation}
Using (\ref{schrodinger-equation}), and its complex conjugate, we have
\begin{equation}
\frac{d \bar{p}_{1}}{d t}=_{S}\left\langle\psi\left|\left(i H p_{1}^{S}+\frac{\partial p_{1}^{S}}{\partial t}-i p_{1}^{S} H\right)\right| \psi\right\rangle_{S}=_{S}\left\langle\psi\left|-i\left[p_{1}^{S}, H\right]\right| \psi\right\rangle_{S}+_{S}\left\langle\psi\left|\frac{\partial p_{1}^{S}}{\partial t}\right| \psi\right\rangle_{S}
\end{equation}
where the last term is equal to zero.\redp{Recall the old NRQM adage that the expectation value of any operator without explicit time dependence that commutes with the Hamiltonian is conserved (its time derivative is zero.)}

\textbf{\redp{The Schrodinger picture states and operators can be transformed to states and operators having different form via what is known as a unitary transformation. The particular unitary transformation (where $U$ is a unitary operator) for this is}}
\begin{qt}
\begin{equation}
U=e^{-i A t} \quad\left(=e^{-i H t / \hbar} \text { in non-natural units }\right)
\end{equation}
where states and operators transform as
\begin{equation}
\begin{array}{ll}
{U^{\dagger}|\psi\rangle_{S}=|\psi\rangle_{H}} & {U^{\dagger} \mathcal{O}^{S} U=\mathcal{O}^{H}} \\
{U|\psi\rangle_{H}=|\psi\rangle_{S}} & {U O^{H} U^{\dagger}=\mathcal{O}^{S}}
\end{array}
\end{equation}
\end{qt}
We find that in the first relation the state is now \redp{time independent} in H.P. 

Taking the time derivative in the second relation, we have:
\begin{equation}
\begin{aligned}
\frac{d}{d t}\left(U^{\dagger} \mathcal{O}^{S} U\right) &=(i H) \underbrace{e^{i A^{\prime}} \mathcal{O}^{S}}_{\mathcal{O}^{H}}+\underbrace{e^{i H_{t}}\left(\frac{\partial \mathcal{O}^{S}}{\partial t}\right) e^{-i H_{l}}}_{\text {defined as } \hat{\partial} O^{H} / \partial t}+\underbrace{e^{i H t} \mathcal{O}^{S} e^{-t H t}}_{\mathcal{O}^{H}}(-i H) \\
&=\frac{d \mathcal{O}^{H}}{d t}=-i\left[\mathcal{O}^{H}, H\right]+\frac{\hat{\partial} \mathcal{O}^{H}}{\partial t}
\end{aligned}
\end{equation}
In this note, $\frac{\hat{\partial} \mathcal{O}^{H}}{\partial t}$ will always be zero. Nonetheless, we see that in the H.P., an operator time derivative can be \textbf{non-zero},\redp{and the operator is time dependent}.
\begin{mybox}
A unitary transformation is called unitary because its operation on (transformation of ) a state vector leaves the magnitude of the state vector unchanged, i.e., the state vector magnitude is multiplied by unity.

A unitary transformation can be thought of as "rotating" a (complex number) state vector in Hilbert space (the complex space where each coordinate axis is an eigenvector) without changing the "length".

Recall, from classical mechanics, that an orthogonal transformation represented by a real matrix $\mathbf{A}$ has an inverse equal to the transpose of that matrix, i.e., $\mathrm{A}^{-1}=\mathrm{A}^{\mathrm{T}}$. In the complex space of state vectors, a unitary transformation $U$ has an analogous form for its inverse, the complex conjugate transpose, i.e., $\underline{U}^{-1}=U^{+}$ and so $U^{\dagger} U=1 .$ The following example may make this clearer.

Do a Taylor expansion of $U=e^{-i H t}$ above about $t$, when $U$ is operating on an energy eigenstate., i.e.,
\begin{equation}
U\left|\psi_{E}\right\rangle= e^{-iHt}\left|\psi_{E}\right\rangle=\left(1-i t H-\frac{1}{2} t^{2} H^{2}+\ldots\right)\left|\psi_{E}\right\rangle= e^{-i E t}\left|\psi_{E}\right\rangle
\end{equation}
\end{mybox}
\textbf{Note:} Although it is common to write $U=e^{-iHt}$,\bluep{it is implied that $H$ (if you think of it as $i\partial/\partial t$) does not act on $t$. To be proper, the $t$ should be placed before the $H$, as we did in the expansion above, but it usually is not done that way.}

\redp{Because $H=H^S$ commutes with itself, $U$ and $U^+$ commute with $H$, so using $\mathcal{O}^S=H^S=H$, we have}
\begin{equation}
    H=H^S=H^H
\end{equation}
and after inserting $UU^+=1$, we find
\begin{equation}
\begin{aligned}
\frac{d \overline{\mathcal{O}}}{d t} &=s\left(\psi\left|U U^{\dagger}\left(-i\left[\mathcal{O}^{S}, H\right]\right) U U^{\dagger}\right| \psi\right\rangle_{s}+_{S}\left\langle\psi\left|U U^{+} \frac{\partial \mathcal{O}^{S}}{\partial t} U U^{\dagger}\right| \psi\right\rangle_{S} \\
&=_{H}\left\langle\psi\left|\left(-i\left[\mathcal{O}^{H}, H\right]\right)\right| \psi\right\rangle_{H}+{}_{H}\left\langle\psi\left|\frac{\hat{\partial} \mathcal{O}^{\mu}}{\partial t}\right| \psi\right\rangle_{H}
\end{aligned}
\end{equation}
\section{Extrapolation to Field Theory}
\textbf{According to the correspondence principle, in the macroscopic limit, our quantum relations must reduce to the usual classical relations.} So the principle provides us with a key part of our method for quantization. That is, in going from classical theory to NRQM, we must take
$$\left\{x^{\prime}, p_{j}\right\}=\delta_{j}^{i} \quad \substack{\longrightarrow\\ \text { 1st quantization }} \quad\left[x^{i}, p_{j}\right]=i \hbar \delta_{j}^{i}$$
For field theory, we have
\begin{qt}
\begin{equation}
\left\{\phi^{r}(\mathrm{x}, t), \pi,(\mathrm{y}, t)\right\}=\delta^{r}{ }_s \delta(\mathrm{x}-\mathrm{y})\substack{\longrightarrow\\ \text{2nd quantization}}\left[\phi^{r}, \pi_{s}\right]=i \hbar \delta^{r}{ }_s \delta(\mathbf{x}-\mathbf{y}) 
\label{2nd-quantization-commutator}
\end{equation}
and
\begin{equation}
    \left[\phi^{r}, \phi^{s}\right]=\left[\pi_{r}, \pi_{s}\right]=0
\end{equation}
\end{qt}
Again, \redp{Quantization is a means for deducing quantum theory from classical theory.}

\section{Appendix: some useful math relations}
First, we have four-velocity of relativity $u^{\mu}$ for an object as:
\begin{equation}
u^{\mu}=\frac{d x^{\mu}}{d \tau}=\frac{d}{d \tau}\left[\begin{array}{ccc}
{x^{0}} & {x^{1}} & {x^{2}}
\end{array}\right]=\left[\begin{array}{cccc}
{u^{0}} & {u^{1}} & {u^{2}} & {u^{3}}
\end{array}\right]
\end{equation}
where
\begin{equation}
u^{i}=\frac{d x^{i}}{d \tau}=\frac{d x^{i}}{\sqrt{1-v^{2} / c^{2}} d t}=\frac{v^{i}}{\sqrt{1-v^{2} / c^{2}}}=\gamma v^{i}
\end{equation}
\begin{equation}
u^{0}=\frac{d x^{0}}{d \tau}=c \frac{d t}{d \tau}=\frac{c}{\sqrt{1-v^{2} / c^{2}}}=\gamma c
\end{equation}
Thus,
\begin{equation}
(u)^{2}=\left|u^{\mu}\right|^{2}=\mu^{\mu} u_{\mu}=\frac{d x^{\mu}}{d \tau} \frac{d x_{\mu}}{d \tau}=\gamma^{2}\left(c^{2}-\left(v^{1}\right)^{2}-\left(v^{2}\right)^{2}-\left(v^{3}\right)^{2}\right)=c^{2}
\end{equation}

The magnitude of the \redp{4-momentum $p^{\mu}=mu^{\mu}$} is  then found from:
\begin{equation}
(p)^{2}=\left|p^{\mu}\right|^{2}=p^{\mu} p_{\mu}=m^{2} u^{\mu} u_{\mu}=m^{2} c^{2} \quad\left(=m^{2} \text { in natural units }\right)
\end{equation}
\bluep{Note that $p^0=\gamma mc=E/c$, where $E$ is relativistic energy, and $p^i$ is relativistic 3-momentum.} Or
\begin{equation}
    \begin{array}{l}
{p^{\mu} p_{\mu}=m^{2} c^{2}=g_{\mu \nu} p^{\mu} p^{\nu}=\left[\begin{array}{cccc}
{E / c} & {p^{1}} & {p^{2}} & {p^{3}}
\end{array}\right]\left[\begin{array}{c}
{E / c} \\
{-p^{1}} \\
{-p^{2}} \\
{-p^{3}}
\end{array}\right]}
\end{array}
\end{equation}

and
\begin{equation}
\frac{E^{2}}{c^{2}}=p^{2}+m^{2} c^{2}
\end{equation}