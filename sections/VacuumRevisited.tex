\chapter{The Vacuum Revisited}
\section{Casimir Plates}
two plates brought close together experience a small attractive force at very small separation distance. This effect was first predicted by Dutch physicists Hendrik B. G. Casimir and Dirk Polder in 1948. The attractive force has been attributed to ZPE, in heuristic and very simple terms, because the the vacuum quantum waves outside the plates presumably exert greater force than the vacuum quantum waves between the plates. However, there are two key things to note:

1) While the Casimir effect can be calculated by assuming ZPE half quanta, the same result can also be calculated another way without using them at all. It thus does not prove their existence, contrary to what is often claimed.

2) In the Casimir calculation that does employ ZPE, the quanta are assumed to be continuously existing standing waves, not particle pairs popping in and out of existence.

\section{Lamb Shift}
The Lamb shift is a small difference between the two energy levels ${}^2 S_{12}$ and ${}^2 P_{12}$ of the hydrogen atom, which according to RQM, should have the same energies. QFT, in its QED form, predicts this shift, and that prediction was one of the great early successes of the theory.

The Lamb shift calculation is long and difficult. It is often described as taking vacuum fluctuations into account in order to obtain the correct result. However, in actuality, \textbf{these "vacuum fluctuations" are really the radiative, or higher order, corrections (extra virtual particles in Feynman diagrams)}. These corrections to the Coulomb potential of the hydrogen atom (in diagrams, extra virtual photons, electrons, and positrons) yield the correct energy levels.

\bluep{The Lamb shift does not prove vacuum pair production/destruction.}